\documentclass[]{article}

\usepackage{listings}

\begin{document}
    Follow the installation instructions on the official Raspberry Pi page in order to install the latest Raspbian.
    In this instruction, I'm using the debian buster based build of 2020/03/20.\\
    Hint: Create an empty file with name "ssh" inside the boot partition of the sd-card.
    This enables ssh access on the RPI.
    You could do that manually as well by connecting a screen and keyboard,
    but since we are going to use the RPI in headless mode,
    that trick saves a lot of time.\\
    Now we're ready to get started!
    Connect the RPI with ethernet and power it up.
    ssh onto it:

    First of all, as you should always do before changing any packages on your linux system, update and upgrade it:
    \begin{lstlisting}
        $ sudo apt-get update
        $ sudo apt-get upgrade
    \end{lstlisting}

    Afterwards, install the editor of your choice - mine is vim:
    \begin{lstlisting}
        $ sudo apt-get install vim
    \end{lstlisting}

    Install a firewall and configure ssh access.
    We'll be using ufw, which provides some basic but still good firewall rules.
    We'll also limit ssh access (find out what exactly that means!)
    \begin{lstlisting}
        $ sudo apt-get install ufw
        $ sudo ufw allow ssh
        $ sudo ufw limit ssh
    \end{lstlisting}

\end{document}